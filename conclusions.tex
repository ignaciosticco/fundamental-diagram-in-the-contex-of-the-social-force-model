\section{\label{conclusions}Conclusions}

Our investigation focused on the fundamental diagram in the context of the social force model. We took a particular case of study (pilgrimage of the entrance to the Jamaraat bridge) to compare the numerical simulation results with the empirical results measured by Helbing and co-workers. We found out that the  social force model in its original fashion is not capable of reproducing the fundamental diagram since the flow increases even for overcrowded scenarios. Using analytical techniques we discovered that either increasing the friction coefficient or increasing the relaxation time may be the key to achieving a non-increasing flow in the congested regime of the fundamental diagram. The second approach was the one implemented by Johansson and a similar idea was introduced in Ref.\cite{parisi2}. We proved that both approaches are equivalent since both of them affect in the same way, the same term of the reduced-in-units equation of motion.\\

The analytical schematic model suggests that the problem could be addressed by modifying the friction coefficient. In light of this, we performed numerical simulations increasing $\kappa$ and attained the fundamental diagram behavior reported in the empirical measurements. \\

When exploring the velocity profile, we found that it has a parabolic shape. Pedestrians reach the maximum velocity in the middle of the corridor while the minimum is by the walls. We found out that once scaled by $v_{max}$ and the corridor width, the velocity profile yields a universal behavior (regardless the width of the corridor).\\

The presence of clustering structures was found to be prompted by the increment of the friction coefficient. Interestingly, increasing the pedestrian-pedestrian friction ($\kappa_i$) or the pedestrian-wall friction coefficient ($\kappa_w$) yields a similar clusterization dependence with density.\\

The phenomena reported in this paper suggests that further research needs to be done regarding the friction coefficient. We propose modeling the pedestrian-wall friction interaction with a different coefficient than the pedestrian-pedestrian friction interaction. We want to stress the fact that studying the friction coefficients may be a critical factor to properly reproduce the dynamics of a massive evacuation under high levels of anxiety. 



