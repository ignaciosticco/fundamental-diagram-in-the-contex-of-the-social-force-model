\section{\label{introduction}Introduction}

By the late '90s and the begining of the century, Helbing and co-workers 
postulated that either the environment and the individuals' own desire affect 
the pedestrians motion in a similar way as forces do with respect to the 
momentum of particles \cite{Helbing1,Helbing4}. This ``social force model'' 
(SFM) nicely bridged the socio-psychological phenomenon of crowds 
behavior to the ``microscopic'' formalism of moving particles. The model 
succeeded at this instance to explain why the crowd evacuation slows down 
as pedestrians try harder to escape fron a dangerous situation 
(\textit{i.e.} ``faster is slower'' effect) \cite{Helbing1,Dorso1,Dorso2}. 
  \\ 

The SFM, in its basic version, was reported to be suitable for a variety of 
panic situations, including the presence of obstacles, or the existence of more 
than one exit \cite{Dorso3,Dorso5}. More sophisticated scenarios required, 
however, a step up implementation \cite{Dorso4,Dorso6,Cornes1}, although 
sustaining the basic model and its parameters.  \\

Some questioning arose on the precise meaning of the social forces. These 
forces stand for the tendency .... \\

