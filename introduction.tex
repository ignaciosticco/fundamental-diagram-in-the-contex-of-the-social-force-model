\section{\label{introduction}Introduction}

By the late '90s and the beginning of the century, Helbing and co-workers 
postulated that either the environment or the individuals' own desire affect 
the pedestrians motion in a similar way as forces do with respect to the 
momentum of particles \cite{Helbing1,Helbing4}. This ``social force model'' 
(SFM) nicely bridged the socio-psychological phenomenon of crowds 
behavior to the ``microscopic'' formalism of moving particles. The model 
succeeded at this instance to explain why the crowd evacuation slows down 
as pedestrians try harder to escape from a dangerous situation 
(\textit{i.e.} ``faster is slower'' effect) \cite{Helbing1,Dorso1,Dorso2}. \\ 

The SFM, in its basic version, was reported to be suitable for describing, 
at least qualitatively, a variety of panic situations, including the presence
of obstacles, or the existence of more than one exit \cite{Dorso3,Dorso5}.
More sophisticated scenarios required, however, a step up implementation
\cite{Dorso4,Dorso6,Cornes1}, although sustaining the basic model and its parameters.  \\

Some questioning arose on the true psychological tendency of the pedestrians to 
stay away from each other. While the social forces accomplish this tendency, it 
attains a somewhat unrealistic ``colliding behavior'' for slowly moving 
pedestrians \cite{Lakoba}. His (her) repulsive tendency is expected to decrease 
as approaching a more crowded environment. The small fall-off length $B=0.08\,$m 
suggested by Helbing in Ref.~\cite{Helbing1} does not completely solve this 
issue. It neither agrees with the fact that pedestrians prefer to keep a 
comfortable $0.5\,$m distance between each other in a moderately crowded 
environment, nor it fits accurately the empirical velocities reported for 
non-panicking crowds \cite{Lakoba}.  \\

Researchers turned back to examine the available data on the velocity and flux 
behavior for different density environments \cite{helbing3,seyfried,seyfried1}. 
Ref.~\cite{helbing3} is a wonderful summary of empirical data from literature, 
and their own data set, acquired from videos of the Muslim pilgrimage in 
Mina-Makkah (2006). They showed from the empirical fundamental diagram (flux $J$ 
versus density $\rho$) that highly dense crowds (seemingly up to $10\,$p/m$^2$) 
do not drive the pedestrians velocity to zero, although the reasons for this 
remain rather obscure.   \\


The high density regime appears to be the most cumbersome one. Caution was 
claimed when (automatically) transferring the usual ``calibrated'' parameters 
of the SFM to this regime. It was argued that the pedestrians' body size 
distribution and the ``situational context'' are somewhat responsible for the 
unexpected departure from these parameters \cite{johansson1,kwak}. But other 
researchers pointed out that this departure actually expresses the lack of a 
mechanism to properly handle the pedestrians' ``required space to move''. Some 
modifications to the basic SFM were then proposed to overcome this difficulty 
\cite{parisi2,seyfried2}. \\

{\color{red} A mechanism allowing an ``increase of the space to move'' (particularly in relaxed/low density situations) is a compelling necessity in the context of the SFM.
But a sharp ``re-calibration'' of the model for high density situations appears not 
to be completely satisfactory \cite{johansson}.} A more ``natural'' way of 
handling this matter requires a deep examination of the current SFM parameters. 
The net-time headway (roughly, the relaxation time) was first examined in 
Ref.~\cite{johansson}. The author sustains the hypothesis that the 
pedestrians net-time headway should increase until there is ``enough space to 
make a step''.  He shows that a density dependent net-time headway is a 
suitable parameter to smartly reproduce the empirical fundamental diagram for 
highly dense crowds \cite{johansson}.  \\ 

Our own examination of the SFM parameters suggests that not only the net-time 
headway, but the friction between pedestrians (and with the walls) can 
reproduce the pattern of the fundamental diagram. Our working hypothesis is 
that friction is the crucial parameter in the dynamics of highly dense crowds. 
We actually sustain the SFM model with no further ``re-calibrations'', but 
with the right friction setting, in order to meet the fundamental diagram 
pattern. \\  

We want to emphasize that although the friction setting appearing in 
Ref.~\cite{Helbing1} is a commonly accepted estimate throughout the  
literature, other values have also been proposed \cite{colombi2017}. We 
intend our setting, however, as an experimental based parameter, suitable for 
high dense crowds. \\

The investigation is organized as follows. We first recall the SFM in 
Section~\ref{sfm}, while including the precise definitions for flux, density 
and clustered structures. Section~\ref{simulations} presents our numerical 
simulations for pedestrians moving through corridors. The corresponding results 
are shown in Section~\ref{results}. Our main conclusions are detailed in the 
closing Section~\ref{conclusions}. A complementary simple model for pedestrians 
moving through a corridor has been included in the Appendix. \\     

