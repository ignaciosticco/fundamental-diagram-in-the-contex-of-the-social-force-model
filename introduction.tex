\section{\label{introduction}Introduction}

By the late '90s and the begining of the century, Helbing and co-workers 
postulated that either the environment and the individuals' own desire affect 
the pedestrians motion in a similar way as forces do with respect to the 
momentum of particles \cite{Helbing1,Helbing4}. This ``social force model'' 
(SFM) nicely bridged the socio-psychological phenomenon of crowds 
behavior to the ``microscopic'' formalism of moving particles. The model 
succeeded at this instance to explain why the crowd evacuation slows down 
as pedestrians try harder to escape fron a dangerous situation 
(\textit{i.e.} ``faster is slower'' effect) \cite{Helbing1,Dorso1,Dorso2}. 
  \\ 

The SFM, in its basic version, was reported to be suitable for a variety of 
panic situations, including the presence of obstacles, or the existence of more 
than one exit \cite{Dorso3,Dorso5}. More sophisticated scenarios required, 
however, a step up implementation \cite{Dorso4,Dorso6,Cornes1}, although 
sustaining the basic model and its parameters.  \\

Some questioning arose, on the true psychological tendency of the pedestrians to 
stay away from each other. While the social forces accomplish this tendency, it 
attains a somewhat unrealistic ``colliding behavior'' for slowly moving 
pedestrians \cite{Lakoba}. His (her) repulsive tendency is expected to decrease 
as approaching a more crowded environment. The small fall-off length $B=0.08\,$m 
suggested by Helbing in Ref.~\cite{Helbing1} does not completely solve this 
issue. It neither agrees with the fact that pedestrians prefer to keep a 
confortable $0.5\,$m distance between each other in a moderately crowded 
environment, nor it fits accurately the empirical velocities reported for 
non-panicking crowds \cite{Lakoba}.  \\

Researchers turned back to examine the available data on the velocity and flux 
behavior for different density environments \cite{helbing3,Seyfried}. 
Ref.~\cite{helbing3} is a wonderful summary of empirical data from literature, 
and their own data set, acquired from videos of the Muslim pilgrimage in 
Mina-Makkah (2006). They showed from the empirical fundamental diagram (flux $J$ 
versus density $\rho$) that highly dense crowds (seemengly up to $10\,$p/m$^2$) 
do not drive the pedestrians velocity to zero, although the reasons for this 
remain rather obscure.   \\

The high density regime appears to be the most cumbersome one. Many authors 
suggested ...   \\