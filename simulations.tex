\section{\label{simulations}Numerical simulations}

The simulations processes were performed on a straight corridor of length $L=$ 28~m and variable width size $w$. We explored different width values ranging from $w=2$~m to $w=40$~m. The corridor had two horizontal walls. One of them was placed at $y=0$ and the other one was placed at $y=w$. The length of each wall was $L$. The pedestrians were modeled as soft spheres of radius $R_i=0.23$~m, this size was fixed according to Ref.~\cite{metric_handbook}. Initially, the individuals were randomly distributed along the corridor with a fixed global density and with random initial velocities, resembling a Gaussian distribution with null mean value. We explored global density values in the range $1<\rho<9$. The number of pedestrians in the simulation was given by the global density and the corridor dimensions chosen in each case. \\

The simulations were supported by LAMMPS molecular dynamics simulator with parallel computing capabilities \cite{plimpton}.
The time integration algorithm followed the velocity Verlet scheme with a time step of $10^{-4}$~s. All the necessary parameters
were set to the same values as in previous works (see Refs. \cite{sticco,Dorso5}), except for the friction coefficient $\kappa$. In this work we use the original value $\kappa=2.4 \times 10^{5}$ and we further utilize $\kappa_i=2.4 \times 10^{6}$ and $\kappa_w=2.4 \times 10^{6}$, being $\kappa_i$ and $\kappa_w$ the pedestrian-pedestrian friction coefficient and the pedestrian-wall friction coefficient respectively. \\

We implemented special modules in C++ for upgrading the LAMMPS capabilities to attain the social force model simulations. We also checked over the LAMMPS output with previous computations (see Refs. \cite{Dorso1, Dorso2,Dorso3, Dorso4,Dorso6}).\\

The desired velocity for each pedestrian $i$ was $\vec{v}_d^{~i}=1$~m/s~$\hat{e}_d^{~i}$, where the target $\hat{e}_d^{~i}$ was set as the $\hat{e}_d^{~i}=(L,y_i)\left \| (L,y_i) \right \|^{-1}$, being $L$ the x-location of the end of the corridor and $y_i$ the y-location corresponding to the ith pedestrian. This produced the pedestrians to move from left to right in an unidirectional flow. Pedestrians that overpassed $x=L$ were re-injected in $x=0$ preserving its current velocity and y-location (\textit{i.e.} periodic boundary conditions). This mechanism was carried out in order to keep
the crowd size unchanged.\\

The measurements were taken once the system reaches the stationary state ($t=30$~s), the configurations of the systems were recorded every 0.05~s, that is, at intervals as short as 10\% of the pedestrian’s relaxation time (see Sec. \ref{sfm}). The recorded magnitudes were the pedestrian’s positions and velocities for each process. We also computed the clusterering structures using a LAMMPS built in function.  
