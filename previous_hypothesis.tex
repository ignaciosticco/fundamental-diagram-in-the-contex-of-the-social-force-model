
{\color{red}
\subsection{Previous Hypothesis}

We stress the fact that our investigation focuses on moving
pedestrians in a highly dense situation (say, the one experienced at
the Jamaraat bridge). As mentioned in Section \ref{introduction} , a deep
examination of the (basic) SFM parameters is required before
proceeding to any extension of the model.\\

Recall from the video analysis of the Muslim pilgrimage in Mina/Makkah
(see Section \ref{introduction}) that high-density flows can turn to a
temporarily interrupted flux, or even to ``turbulent" flux as people
start pushing to gain space \cite{helbing3}. This unexpected behavior can not be
reproduced by the (basic) SFM, seemingly because of ``an
underestimation of the local interactions triggered by high densities"
\cite{yu1}, or, the absence of a ``delayed reaction in cases of unexpected
behaviors" \cite{johansson}. Both statements are currently working hypotheses since
experimental data (specifically, measurements of pedestrian flux and
densities) does not ``provide any insight into the mechanisms and
dynamics behind the pedestrians interactions and behaviors" \cite{johansson}.\\

Researchers propose a ``re-calibration" of the (basic) SFM, in order to
attain ``stop-and-go" or ``turbulent" flows for highly dense crowds
\cite{yu1,johansson}. Presumably, this kind of instabilities within the crowd prevent
people from stopping at extremely high densities. The intended
``re-calibration" consist on either enhancing the (local) social
interactions or increasing the net-time headway (roughly, the
relaxation time) for the high density regime. Both extensions,
however, may not exclude other possibilities involving not only
individual motion, but collective (mass) motion \cite{helbing3}. Researchers
further point out that the relevance of physical contact in extremely
dense crowds may suppose that a somewhat commonality with granular
media exists \cite{helbing3}.\\

We do not intend to reproduce ``turbulent" dynamics, but show that the
experimental data (say, the flux-density diagram) can be modeled under
quasi-stationary conditions in the high density regime. Our starting
point is the re-examination of the (basic) SFM. We hypothesize that two
control parameters, $\mathcal{A}$ and $\mathcal{K}$ (see appendix \ref{appendix_1}) may handle
the collective motion in dense crowds, although ``crowd turbulence" or
``crowd panic" may not be present. We presume that physical contact is
a key feature in dense crowds, despite the fact that other issues may
also contribute to the flow reduction \cite{johansson1}. However, the latter could
be satisfactorily omitted in past research \cite{johansson}, and thus, we will not
attempt to introduce further extensions to the (basic) SFM for the
sake of simplicity.\\

The former ``re-calibrations" accomplish the socio-psychological
response of the crowd to ``gain more space" (by either enhancing local
interactions or performing a delayed reaction). We are aware that
crowds may respond differently in many situations (see Ref.~\cite{drury1}). Our working
hypothesis, however, does not focus on the crowd socio-psychological
response, but on the physical contact among pedestrians (and the
walls). The socio-psychological attitude of the pedestrians will be
assumed to remain fixed along the simulations (with the anxiety level
limited to $v_d=1\,$m/s). Notice that these assumptions ignore the
possibly of ``crowd turbulence" or ``crowd panic", which are out of the
scope of our ``re-calibration" procedure.\\


Our investigation appears somewhat restricted to the (almost)
uni-direction flow inspired in the Muslim pilgrimage in Mina/Makkah.
This means that the following ``re-calibration" results hold for
corridor-like situations, and are not intended to be (automatically)
translated to bottleneck situations. Neither can be extended to other
boundary conditions (say, no limiting walls) since the boundary is a
key feature of collective motion. Nevertheless, our results accomplish
the available data on the Hajj pilgrimages \cite{helbing3,lohner1}.

}
