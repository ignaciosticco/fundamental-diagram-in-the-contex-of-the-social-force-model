\begin{abstract}

The fundamental diagram of pedestrian dynamics gives the relation between the density and the flow in a specific enclosure. It is characterized by two distinctive behaviors: the free-flow regime (for low densities) and the congested regime (for high densities). In the former, the flow increases as density increases, while on the latter, the flow does not increase with density. In this work we perform numerical simulations of the pilgrimage of the entrance to the Jamarat bridge (pedestrians walking along a straight corridor) and compare the flow-density measurements with the empirical measurements made by Helbing et al. We show that under high density conditions, the Social Force Model is not capable of reproducing the fundamental diagram reported in empirical measurements. We use analytical techniques and numerical simulations to prove that with an appropriate modification of the friction coefficient it is possible to attain the qualitative behavior of the expected flow-density relation. Other authors proposed a modification of the relaxation time in order to address this problem. We unveil the fact that our approach is analogous to theirs, since both affect the same term of the reduced-in-units equation of motion. We inspect how the friction modification affects the pedestrian clustering structures present in the transition from the free-flow regime to the congested regime. We also discover that the speed profile normalized by width and maximum velocity yield a universal behavior regardless the corridor dimensions. 

\end{abstract}